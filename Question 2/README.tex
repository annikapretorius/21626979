\documentclass[12pt,preprint, authoryear]{elsarticle}

\usepackage{lmodern}
%%%% My spacing
\usepackage{setspace}
\setstretch{1.2}
\DeclareMathSizes{12}{14}{10}{10}

% Wrap around which gives all figures included the [H] command, or places it "here". This can be tedious to code in Rmarkdown.
\usepackage{float}
\let\origfigure\figure
\let\endorigfigure\endfigure
\renewenvironment{figure}[1][2] {
    \expandafter\origfigure\expandafter[H]
} {
    \endorigfigure
}

\let\origtable\table
\let\endorigtable\endtable
\renewenvironment{table}[1][2] {
    \expandafter\origtable\expandafter[H]
} {
    \endorigtable
}


\usepackage{ifxetex,ifluatex}
\usepackage{fixltx2e} % provides \textsubscript
\ifnum 0\ifxetex 1\fi\ifluatex 1\fi=0 % if pdftex
  \usepackage[T1]{fontenc}
  \usepackage[utf8]{inputenc}
\else % if luatex or xelatex
  \ifxetex
    \usepackage{mathspec}
    \usepackage{xltxtra,xunicode}
  \else
    \usepackage{fontspec}
  \fi
  \defaultfontfeatures{Mapping=tex-text,Scale=MatchLowercase}
  \newcommand{\euro}{€}
\fi

\usepackage{amssymb, amsmath, amsthm, amsfonts}

\def\bibsection{\section*{References}} %%% Make "References" appear before bibliography


\usepackage[round]{natbib}

\usepackage{longtable}
\usepackage[margin=2.3cm,bottom=2cm,top=2.5cm, includefoot]{geometry}
\usepackage{fancyhdr}
\usepackage[bottom, hang, flushmargin]{footmisc}
\usepackage{graphicx}
\numberwithin{equation}{section}
\numberwithin{figure}{section}
\numberwithin{table}{section}
\setlength{\parindent}{0cm}
\setlength{\parskip}{1.3ex plus 0.5ex minus 0.3ex}
\usepackage{textcomp}
\renewcommand{\headrulewidth}{0.2pt}
\renewcommand{\footrulewidth}{0.3pt}

\usepackage{array}
\newcolumntype{x}[1]{>{\centering\arraybackslash\hspace{0pt}}p{#1}}

%%%%  Remove the "preprint submitted to" part. Don't worry about this either, it just looks better without it:
\makeatletter
\def\ps@pprintTitle{%
  \let\@oddhead\@empty
  \let\@evenhead\@empty
  \let\@oddfoot\@empty
  \let\@evenfoot\@oddfoot
}
\makeatother

 \def\tightlist{} % This allows for subbullets!

\usepackage{hyperref}
\hypersetup{breaklinks=true,
            bookmarks=true,
            colorlinks=true,
            citecolor=blue,
            urlcolor=blue,
            linkcolor=blue,
            pdfborder={0 0 0}}


% The following packages allow huxtable to work:
\usepackage{siunitx}
\usepackage{multirow}
\usepackage{hhline}
\usepackage{calc}
\usepackage{tabularx}
\usepackage{booktabs}
\usepackage{caption}


\newenvironment{columns}[1][]{}{}

\newenvironment{column}[1]{\begin{minipage}{#1}\ignorespaces}{%
\end{minipage}
\ifhmode\unskip\fi
\aftergroup\useignorespacesandallpars}

\def\useignorespacesandallpars#1\ignorespaces\fi{%
#1\fi\ignorespacesandallpars}

\makeatletter
\def\ignorespacesandallpars{%
  \@ifnextchar\par
    {\expandafter\ignorespacesandallpars\@gobble}%
    {}%
}
\makeatother

\newlength{\cslhangindent}
\setlength{\cslhangindent}{1.5em}
\newenvironment{CSLReferences}%
  {\setlength{\parindent}{0pt}%
  \everypar{\setlength{\hangindent}{\cslhangindent}}\ignorespaces}%
  {\par}


\urlstyle{same}  % don't use monospace font for urls
\setlength{\parindent}{0pt}
\setlength{\parskip}{6pt plus 2pt minus 1pt}
\setlength{\emergencystretch}{3em}  % prevent overfull lines
\setcounter{secnumdepth}{5}

%%% Use protect on footnotes to avoid problems with footnotes in titles
\let\rmarkdownfootnote\footnote%
\def\footnote{\protect\rmarkdownfootnote}
\IfFileExists{upquote.sty}{\usepackage{upquote}}{}

%%% Include extra packages specified by user

%%% Hard setting column skips for reports - this ensures greater consistency and control over the length settings in the document.
%% page layout
%% paragraphs
\setlength{\baselineskip}{12pt plus 0pt minus 0pt}
\setlength{\parskip}{12pt plus 0pt minus 0pt}
\setlength{\parindent}{0pt plus 0pt minus 0pt}
%% floats
\setlength{\floatsep}{12pt plus 0 pt minus 0pt}
\setlength{\textfloatsep}{20pt plus 0pt minus 0pt}
\setlength{\intextsep}{14pt plus 0pt minus 0pt}
\setlength{\dbltextfloatsep}{20pt plus 0pt minus 0pt}
\setlength{\dblfloatsep}{14pt plus 0pt minus 0pt}
%% maths
\setlength{\abovedisplayskip}{12pt plus 0pt minus 0pt}
\setlength{\belowdisplayskip}{12pt plus 0pt minus 0pt}
%% lists
\setlength{\topsep}{10pt plus 0pt minus 0pt}
\setlength{\partopsep}{3pt plus 0pt minus 0pt}
\setlength{\itemsep}{5pt plus 0pt minus 0pt}
\setlength{\labelsep}{8mm plus 0mm minus 0mm}
\setlength{\parsep}{\the\parskip}
\setlength{\listparindent}{\the\parindent}
%% verbatim
\setlength{\fboxsep}{5pt plus 0pt minus 0pt}



\begin{document}



\begin{frontmatter}  %

\title{The Longevity and Musical Progression of Coldplay versus
Metallica}

% Set to FALSE if wanting to remove title (for submission)




\author[Add1]{Annika Pretorius (21626979)\footnote{\textbf{Contributions:}
  \newline \emph{The authors would like to thank Spotify and the
  Billboard Top 100 for the data utilized within the following
  investigation.}}}
\ead{}





\address[Add1]{Stellenbosch University}

\cortext[cor]{Corresponding author: Annika Pretorius
(21626979)\footnote{\textbf{Contributions:} \newline \emph{The authors
  would like to thank Spotify and the Billboard Top 100 for the data
  utilized within the following investigation.}}}

\begin{abstract}
\small{
This article aims to discuss the longevity and musical progression of
two famous bands: Coldplay and Metallica
}
\end{abstract}

\vspace{1cm}





\vspace{0.5cm}

\end{frontmatter}



%________________________
% Header and Footers
%%%%%%%%%%%%%%%%%%%%%%%%%%%%%%%%%
\pagestyle{fancy}
\chead{}
\rhead{}
\lfoot{}
\rfoot{\footnotesize Page \thepage}
\lhead{}
%\rfoot{\footnotesize Page \thepage } % "e.g. Page 2"
\cfoot{}

%\setlength\headheight{30pt}
%%%%%%%%%%%%%%%%%%%%%%%%%%%%%%%%%
%________________________

\headsep 35pt % So that header does not go over title




\hypertarget{an-analysis-of-the-music-industry}{%
\section{\texorpdfstring{An Analysis of the Music Industry
\label{An Analysis of the Music Industry}}{An Analysis of the Music Industry }}\label{an-analysis-of-the-music-industry}}

Coldplay and Metallica are two iconic bands that have left an indelible
mark on the music industry. Coldplay, known for its atmospheric
soundscapes and introspective lyrics, has captivated audiences worldwide
since its debut in the early 2000s. In contrast, Metallica's powerful
riffs and intense performances have made them a staple in the heavy
metal genre since the 1980s. This project aims to compare the
performance of these two bands using data from the Billboard Top 100 and
Spotify, providing insights into their popularity and influence over
time.

The analysis focuses on several key variables within the datasets. From
the Billboard Top 100, we examine metrics such as the number of songs
each band has placed in the Top 100, the peak rank achieved by these
songs, and the total number of weeks they have remained on the chart.
These metrics offer a comprehensive view of each band's commercial
success and longevity in the music charts. On Spotify, we analyze the
number of streams and popularity scores of their songs, providing a
modern perspective on their relevance in the current digital music
landscape. By comparing these variables, we can gain a deeper
understanding of how Coldplay and Metallica have evolved over time and
how their music continues to resonate with audiences today.

\hypertarget{comparisons-of-the-bands-popularity}{%
\section{\texorpdfstring{Comparisons of the Bands
Popularity\label{Comparisons of the Bands Popularity}}{Comparisons of the Bands Popularity}}\label{comparisons-of-the-bands-popularity}}

Coldplay and Metallica, though distinct in their musical styles, both
enjoy significant popularity across their albums. Popularity in the
figures below measure the popularity of the albums ranging form 0 to
100. The Data source from which this was taken is from spotify

\begin{figure}

{\centering \includegraphics{README_files/figure-latex/unnamed-chunk-1-1} 

}

\caption{Coldplay's Popularity per Album.\label{Figure1}}\label{fig:unnamed-chunk-1}
\end{figure}

\begin{figure}

{\centering \includegraphics{README_files/figure-latex/unnamed-chunk-2-1} 

}

\caption{Metallica's Popularity per Album.\label{Figure2}}\label{fig:unnamed-chunk-2}
\end{figure}

As you can see from \ref{Figure2} and \ref{Figure2}, Coldplay and
Metallica, despite having very different musical styles, both enjoy
massive popularity across their albums. However as seen in \ref{Figure3}
below, Coldplay is more popular overall than Metallica.

\begin{figure}

{\centering \includegraphics{README_files/figure-latex/unnamed-chunk-3-1} 

}

\caption{Comparing both Band's Popularity\label{Figure3}}\label{fig:unnamed-chunk-3}
\end{figure}

\hypertarget{density-plot-of-popularity}{%
\section{Density Plot of Popularity:}\label{density-plot-of-popularity}}

The density plot reveals the distribution of song popularity for
Coldplay and Metallica. Coldplay's songs tend to have a more uniform
distribution with a slight peak, indicating a consistent level of
popularity across their catalog. Metallica, on the other hand, shows a
more pronounced peak, suggesting that a significant number of their
songs reach higher popularity scores. This contrast highlights how
Metallica has a core set of extremely popular tracks, while Coldplay's
popularity is more evenly spread across their discography. This plot
helps us understand the different ways in which each band's music
resonates with their listeners.

\begin{figure}

{\centering \includegraphics{README_files/figure-latex/unnamed-chunk-4-1} 

}

\caption{Density Plot of Popularity\label{Figure4}}\label{fig:unnamed-chunk-4}
\end{figure}

\hypertarget{scatter-plot-of-energy-vs-danceability}{%
\section{Scatter Plot of Energy vs
Danceability:}\label{scatter-plot-of-energy-vs-danceability}}

The scatter plot of energy versus danceability, \ref{Figure5}, offers an
insightful comparison between Coldplay and Metallica's musical styles.
Coldplay's songs generally cluster towards the higher end of
danceability, reflecting their melodic and rhythmic compositions
designed to be both catchy and uplifting. In contrast, Metallica's
tracks tend to score higher on energy but lower on danceability,
showcasing their intense sound that's characteristic of heavy metal.
This scatter plot effectively illustrates the distinct musical
signatures of each band, highlighting Coldplay's tendency towards more
danceable tunes and Metallica's focus on high-energy, powerful
performances.

\begin{figure}

{\centering \includegraphics{README_files/figure-latex/unnamed-chunk-5-1} 

}

\caption{Energy versus Danceability\label{Figure5}}\label{fig:unnamed-chunk-5}
\end{figure}

\hypertarget{analysing-the-billboard-top-100}{%
\section{Analysing the Billboard Top
100}\label{analysing-the-billboard-top-100}}

Coldplay and Metallica have both spent a lot of time on the Billboard
Top 100, showing their lasting popularity. Coldplay's songs, with their
emotional and melodic style, have kept them on the charts for many
weeks. Metallica's powerful and energetic tracks have also secured them
a strong presence, especially in the rock and metal genres. The bar plot
below clearly shows that while both bands have had substantial success,
their chart performance underscores their ability to engage listeners
and sustain their popularity over extended periods. This analysis
provides a compelling visual representation of the lasting impact both
bands have had on the music industry.

\begin{figure}

{\centering \includegraphics{README_files/figure-latex/unnamed-chunk-6-1} 

}

\caption{The Billboard 100s Results for both bands\label{Figure6}}\label{fig:unnamed-chunk-6}
\end{figure}

\newpage

\hypertarget{conclusion}{%
\section{Conclusion}\label{conclusion}}

In conclusion, Coldplay and Metallica have demonstrated remarkable
longevity and musical progression over the years. Coldplay's ability to
evolve their sound while maintaining their signature emotional and
melodic style has kept them relevant and popular. Similarly, Metallica's
consistent delivery of powerful, energetic music has solidified their
place in the heavy metal genre. Both bands have shown that they can
adapt and thrive in the ever-changing music industry, securing a lasting
legacy through their continuous presence on the charts and streaming
platforms.

\bibliography{Tex/ref}





\end{document}
