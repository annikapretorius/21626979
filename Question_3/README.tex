\documentclass[12pt,preprint, authoryear]{elsarticle}

\usepackage{lmodern}
%%%% My spacing
\usepackage{setspace}
\setstretch{1.2}
\DeclareMathSizes{12}{14}{10}{10}

% Wrap around which gives all figures included the [H] command, or places it "here". This can be tedious to code in Rmarkdown.
\usepackage{float}
\let\origfigure\figure
\let\endorigfigure\endfigure
\renewenvironment{figure}[1][2] {
    \expandafter\origfigure\expandafter[H]
} {
    \endorigfigure
}

\let\origtable\table
\let\endorigtable\endtable
\renewenvironment{table}[1][2] {
    \expandafter\origtable\expandafter[H]
} {
    \endorigtable
}


\usepackage{ifxetex,ifluatex}
\usepackage{fixltx2e} % provides \textsubscript
\ifnum 0\ifxetex 1\fi\ifluatex 1\fi=0 % if pdftex
  \usepackage[T1]{fontenc}
  \usepackage[utf8]{inputenc}
\else % if luatex or xelatex
  \ifxetex
    \usepackage{mathspec}
    \usepackage{xltxtra,xunicode}
  \else
    \usepackage{fontspec}
  \fi
  \defaultfontfeatures{Mapping=tex-text,Scale=MatchLowercase}
  \newcommand{\euro}{€}
\fi

\usepackage{amssymb, amsmath, amsthm, amsfonts}

\def\bibsection{\section*{References}} %%% Make "References" appear before bibliography


\usepackage[round]{natbib}

\usepackage{longtable}
\usepackage[margin=2.3cm,bottom=2cm,top=2.5cm, includefoot]{geometry}
\usepackage{fancyhdr}
\usepackage[bottom, hang, flushmargin]{footmisc}
\usepackage{graphicx}
\numberwithin{equation}{section}
\numberwithin{figure}{section}
\numberwithin{table}{section}
\setlength{\parindent}{0cm}
\setlength{\parskip}{1.3ex plus 0.5ex minus 0.3ex}
\usepackage{textcomp}
\renewcommand{\headrulewidth}{0.2pt}
\renewcommand{\footrulewidth}{0.3pt}

\usepackage{array}
\newcolumntype{x}[1]{>{\centering\arraybackslash\hspace{0pt}}p{#1}}

%%%%  Remove the "preprint submitted to" part. Don't worry about this either, it just looks better without it:

 \def\tightlist{} % This allows for subbullets!

\usepackage{hyperref}
\hypersetup{breaklinks=true,
            bookmarks=true,
            colorlinks=true,
            citecolor=blue,
            urlcolor=blue,
            linkcolor=blue,
            pdfborder={0 0 0}}


% The following packages allow huxtable to work:
\usepackage{siunitx}
\usepackage{multirow}
\usepackage{hhline}
\usepackage{calc}
\usepackage{tabularx}
\usepackage{booktabs}
\usepackage{caption}


\newenvironment{columns}[1][]{}{}

\newenvironment{column}[1]{\begin{minipage}{#1}\ignorespaces}{%
\end{minipage}
\ifhmode\unskip\fi
\aftergroup\useignorespacesandallpars}

\def\useignorespacesandallpars#1\ignorespaces\fi{%
#1\fi\ignorespacesandallpars}

\makeatletter
\def\ignorespacesandallpars{%
  \@ifnextchar\par
    {\expandafter\ignorespacesandallpars\@gobble}%
    {}%
}
\makeatother

\newlength{\cslhangindent}
\setlength{\cslhangindent}{1.5em}
\newenvironment{CSLReferences}%
  {\setlength{\parindent}{0pt}%
  \everypar{\setlength{\hangindent}{\cslhangindent}}\ignorespaces}%
  {\par}


\urlstyle{same}  % don't use monospace font for urls
\setlength{\parindent}{0pt}
\setlength{\parskip}{6pt plus 2pt minus 1pt}
\setlength{\emergencystretch}{3em}  % prevent overfull lines
\setcounter{secnumdepth}{5}

%%% Use protect on footnotes to avoid problems with footnotes in titles
\let\rmarkdownfootnote\footnote%
\def\footnote{\protect\rmarkdownfootnote}
\IfFileExists{upquote.sty}{\usepackage{upquote}}{}

%%% Include extra packages specified by user
\usepackage{float}\usepackage{booktabs}
\usepackage{longtable}
\usepackage{array}
\usepackage{multirow}
\usepackage{wrapfig}
\usepackage{float}
\usepackage{colortbl}
\usepackage{pdflscape}
\usepackage{tabu}
\usepackage{threeparttable}
\usepackage{threeparttablex}
\usepackage[normalem]{ulem}
\usepackage{makecell}
\usepackage{xcolor}

%%% Hard setting column skips for reports - this ensures greater consistency and control over the length settings in the document.
%% page layout
%% paragraphs
\setlength{\baselineskip}{12pt plus 0pt minus 0pt}
\setlength{\parskip}{12pt plus 0pt minus 0pt}
\setlength{\parindent}{0pt plus 0pt minus 0pt}
%% floats
\setlength{\floatsep}{12pt plus 0 pt minus 0pt}
\setlength{\textfloatsep}{20pt plus 0pt minus 0pt}
\setlength{\intextsep}{14pt plus 0pt minus 0pt}
\setlength{\dbltextfloatsep}{20pt plus 0pt minus 0pt}
\setlength{\dblfloatsep}{14pt plus 0pt minus 0pt}
%% maths
\setlength{\abovedisplayskip}{12pt plus 0pt minus 0pt}
\setlength{\belowdisplayskip}{12pt plus 0pt minus 0pt}
%% lists
\setlength{\topsep}{10pt plus 0pt minus 0pt}
\setlength{\partopsep}{3pt plus 0pt minus 0pt}
\setlength{\itemsep}{5pt plus 0pt minus 0pt}
\setlength{\labelsep}{8mm plus 0mm minus 0mm}
\setlength{\parsep}{\the\parskip}
\setlength{\listparindent}{\the\parindent}
%% verbatim
\setlength{\fboxsep}{5pt plus 0pt minus 0pt}



\begin{document}



\begin{frontmatter}  %

\title{An Analysis of the Russia-Ukraine Conflict and the Topic of
Country Aid}

% Set to FALSE if wanting to remove title (for submission)




\author[Add1]{Annika Pretorius (21626979)\footnote{\textbf{Contributions:}
  \newline \emph{The producer of ``From Russia With No Love'' for the
  data provided.}}}
\ead{}





\address[Add1]{Australian News Presenter}


\begin{abstract}
\small{
This article aims to discuss whether European countries have aided
Ukraine in the war against Russia.
}
\end{abstract}

\vspace{1cm}





\vspace{0.5cm}

\end{frontmatter}



%________________________
% Header and Footers
%%%%%%%%%%%%%%%%%%%%%%%%%%%%%%%%%
\pagestyle{fancy}
\chead{}
\rhead{}
\lfoot{}
\rfoot{\footnotesize Page \thepage}
\lhead{}
%\rfoot{\footnotesize Page \thepage } % "e.g. Page 2"
\cfoot{}

%\setlength\headheight{30pt}
%%%%%%%%%%%%%%%%%%%%%%%%%%%%%%%%%
%________________________

\headsep 35pt % So that header does not go over title




\hypertarget{introduction}{%
\section{Introduction}\label{introduction}}

\begin{itemize}
\tightlist
\item
  \textbf{Background of the Russia-Ukraine Conflict}

  \begin{itemize}
  \item
    The conflict between Russia and Ukraine escalated in 2014 with
    Russia's annexation of Crimea and support for separatist movements
    in Eastern Ukraine. Since Russia's full-scale invasion in 2022, its
    war against Ukraine has had a disastrous impact on civilian life,
    killing thousands of civilians, injuring many thousands more, and
    destroying civilian property and infrastructure.
  \item
    The conflict has led to significant humanitarian crises and
    geopolitical tensions, impacting international relations and
    security dynamics.
  \end{itemize}
\item
  \textbf{European Union's Response}

  \begin{itemize}
  \tightlist
  \item
    The EU has played a crucial role in providing support to Ukraine,
    both through individual member states and collective efforts.
  \item
    Financial aid and commitments have been made to support Ukraine's
    military, humanitarian, and economic needs.
  \end{itemize}
\item
  \textbf{Purpose of the Analysis}

  \begin{itemize}
  \tightlist
  \item
    This analysis aims to evaluate the extent of aid provided by EU
    countries to Ukraine.
  \item
    The focus is on financial allocations and commitments, with a
    breakdown into different types of aid such as military and
    humanitarian support.
  \end{itemize}
\item
  \textbf{Key Questions Addressed}

  \begin{itemize}
  \tightlist
  \item
    Which EU countries are the top contributors in terms of financial
    allocations and commitments to Ukraine and which are the lowest?
  \item
    How do the types of aid (military vs.~humanitarian) vary among the
    top contributing countries?
  \end{itemize}
\item
  \textbf{Significance of the Analysis}

  \begin{itemize}
  \tightlist
  \item
    Understanding the support provided by EU countries helps gauge the
    international community's efforts in assisting Ukraine.
  \item
    The analysis sheds light on the strategic and humanitarian
    priorities of the EU in addressing the ongoing conflict and its
    repercussions.
  \end{itemize}
\end{itemize}

\hypertarget{the-difference-between-financial-allocations-and-financial-commitments}{%
\section{The Difference between Financial Allocations and Financial
Commitments}\label{the-difference-between-financial-allocations-and-financial-commitments}}

\begin{itemize}
\item
  Financial Allocation refers to the funds that have been officially
  designated or set aside for a specific purpose or project. In this
  case, it means the money that has been allocated by EU countries to
  support Ukraine.An EU country might allocate €1 billion for
  humanitarian aid to Ukraine, which means this amount is reserved in
  their budget for this purpose.
\item
  Financial Commitment refers to the amounts that have been pledged or
  promised to be delivered in support of Ukraine. These are the funds
  that countries have committed to providing, which might be part of
  international agreements or announcements. An EU country might commit
  €500 million to Ukraine over the next two years for military aid,
  which means they promise to provide this amount, but it might be
  delivered in installments.
\item
  Financial Allocations are usually immediate and part of a current
  budget. These funds have been set aside and are ready for
  disbursement. Financial Commitments may span over multiple years and
  indicate future support. These are pledges made to provide aid over a
  longer period and may require further authorization or legislative
  approval before they can be fully disbursed.
\item
  Financial Allocations have an immediate impact as they are part of
  active budgets and can be disbursed quickly. Financial Commitments
  impact planning and international relations, signaling long-term
  support. They indicate a commitment to future aid, which can influence
  diplomatic and strategic planning.
\end{itemize}

\hypertarget{the-top-5-european-union-allocators-of-financial-aid-to-the-ukraine}{%
\section{The Top 5 European Union Allocators of Financial Aid to the
Ukraine}\label{the-top-5-european-union-allocators-of-financial-aid-to-the-ukraine}}

\begin{itemize}
\tightlist
\item
  \textbf{EU (Commission and Council)}:

  \begin{itemize}
  \tightlist
  \item
    Stands out as the largest contributor, with a total allocation of
    \$43.83 billion.
  \item
    Includes significant military support totaling \$34.38 billion.
  \item
    Notable humanitarian allocation of \$10.49 billion.
  \end{itemize}
\item
  \textbf{Estonia}:

  \begin{itemize}
  \tightlist
  \item
    Second-largest contributor with \$29.66 billion.
  \item
    Primarily focused on humanitarian aid, highlighting its significant
    commitment to supporting Ukraine.
  \end{itemize}
\item
  \textbf{Other Notable Contributors}:

  \begin{itemize}
  \tightlist
  \item
    Italy, Bulgaria, France, and Finland have also contributed various
    amounts.
  \item
    France provides the highest military allocation among them at
    \$10.75 billion.
  \end{itemize}
\item
  \textbf{Total Bilateral Allocation}:

  \begin{itemize}
  \tightlist
  \item
    Aggregates these contributions, demonstrating the comprehensive
    financial support provided directly by each country.
  \end{itemize}
\item
  \textbf{Overall Illustration}:

  \begin{itemize}
  \tightlist
  \item
    This table underscores the critical role played by individual EU
    countries in aiding Ukraine.
  \item
    Highlights both military and humanitarian efforts.
  \end{itemize}
\end{itemize}

\begingroup\fontsize{12pt}{13pt}\selectfont
\begin{longtable}{lrrr}
\caption{Summary of EU Aid to Ukraine \label{tab1}} \\ 
  \toprule
Country & Total\_Allocation & Military\_Allocation & Total\_Bilateral\_Allocation \\ 
  \midrule
EU(Commission and Council) & 43.83 & 34.38 & 88.71 \\ 
  Estonia & 29.66 & 0.00 & 32.11 \\ 
  Italy & 5.18 & 0.07 & 6.34 \\ 
  Bulgaria & 3.70 & 2.07 & 6.16 \\ 
  France & 1.51 & 10.75 & 15.52 \\ 
  Finland & 0.86 & 2.88 & 4.13 \\ 
   \bottomrule
\end{longtable}
\endgroup

Table 2.1 above presents a detailed summary of the financial allocations
made by EU countries to Ukraine, with the first column a reference to
the total amount allocated by the EU.

\begin{figure}

{\centering \includegraphics{README_files/figure-latex/unnamed-chunk-1-1} 

}

\caption{Top 10 EU Countries Humanitarian Allocations to Ukraine\label{Figure1}}\label{fig:unnamed-chunk-1}
\end{figure}

\hypertarget{the-top-5-european-countries-in-terms-of-financial-commitment}{%
\section{The Top 5 European Countries in terms of Financial
Commitment}\label{the-top-5-european-countries-in-terms-of-financial-commitment}}

\begin{itemize}
\tightlist
\item
  \textbf{EU (Commission and Council)}:

  \begin{itemize}
  \tightlist
  \item
    Leads with a total commitment of \$71.04 billion.
  \item
    Indicates long-term financial support with a significant military
    commitment of \$56.11 billion.
  \end{itemize}
\item
  \textbf{Germany}:

  \begin{itemize}
  \tightlist
  \item
    Second-largest contributor with a commitment of \$24.71 billion.
  \item
    Includes \$19.94 billion for military aid, reflecting its strategic
    support for Ukraine's defense capabilities.
  \end{itemize}
\item
  \textbf{Other Notable Commitments}:

  \begin{itemize}
  \tightlist
  \item
    The Netherlands, Poland, France, and Austria also make notable
    commitments.
  \item
    Military commitments form a substantial part of their total
    contributions.
  \end{itemize}
\item
  \textbf{Overall Illustration}:

  \begin{itemize}
  \tightlist
  \item
    This table illustrates the pledged support from various EU
    countries.
  \item
    Showcases their readiness to provide ongoing financial and military
    assistance to Ukraine.
  \item
    Reflects the strategic and humanitarian priorities of the EU
    countries in supporting Ukraine amidst ongoing challenges.
  \end{itemize}
\end{itemize}

\begingroup\fontsize{11pt}{12pt}\selectfont
\begin{longtable}{lrrr}
\caption{Summary of EU Commitments to Ukraine \label{tab2}} \\ 
  \toprule
Country & Total\_Commitment & Military\_Commitment & Total\_Bilateral\_Commitment \\ 
  \midrule
EU(Commission and Council) & 7.43 & 56.11 & 71.04 \\ 
  Germany & 1.51 & 19.94 & 24.71 \\ 
  Netherlands & 1.13 & 4.76 & 6.66 \\ 
  Poland & 1.01 & 3.21 & 4.63 \\ 
  France & 0.86 & 6.05 & 7.29 \\ 
  Austria & 0.73 & 0.00 & 0.86 \\ 
   \bottomrule
\end{longtable}
\endgroup

Table 3.1 above summarizes the financial commitments made by EU
countries to Ukraine, emphasizing their pledged support.

\begin{figure}

{\centering \includegraphics{README_files/figure-latex/unnamed-chunk-2-1} 

}

\caption{Top 10 EU Countries Humanitarian Commitments to Ukraine\label{Figure2}}\label{fig:unnamed-chunk-2}
\end{figure}

\hypertarget{lowest-5-european-countries-in-terms-of-financial-allocations}{%
\section{Lowest 5 European Countries in terms of Financial
Allocations}\label{lowest-5-european-countries-in-terms-of-financial-allocations}}

\begingroup\fontsize{12pt}{13pt}\selectfont
\begin{longtable}{lrrr}
\caption{Summary of EU Aid to Ukraine \label{tab1}} \\ 
  \toprule
Country & Total\_Allocation & Military\_Allocation & Total\_Bilateral\_Allocation \\ 
  \midrule
Belgium & 0.00 & 0.26 & 0.26 \\ 
  Cyprus & 0.00 & 1.38 & 1.44 \\ 
  Denmark & 0.00 & 0.52 & 0.58 \\ 
  Germany & 0.00 & 0.14 & 0.14 \\ 
  Greece & 0.00 & 0.00 & 0.06 \\ 
   \bottomrule
\end{longtable}
\endgroup

\hypertarget{lowest-5-european-countries-to-financially-commit}{%
\section{Lowest 5 European Countries to Financially
Commit}\label{lowest-5-european-countries-to-financially-commit}}

\begingroup\fontsize{11pt}{12pt}\selectfont
\begin{longtable}{lrrr}
\caption{Summary of EU Commitments to Ukraine \label{tab2}} \\ 
  \toprule
Country & Total\_Commitment & Military\_Commitment & Total\_Bilateral\_Commitment \\ 
  \midrule
Bulgaria & 0.00 & 0.26 & 0.26 \\ 
  Croatia & 0.00 & 0.20 & 0.30 \\ 
  Czech Republic & 0.00 & 1.38 & 1.45 \\ 
  Estonia & 0.00 & 0.95 & 1.14 \\ 
  Greece & 0.00 & 0.14 & 0.14 \\ 
   \bottomrule
\end{longtable}
\endgroup

\begin{itemize}
\tightlist
\item
  \textbf{General Observations}:

  \begin{itemize}
  \tightlist
  \item
    Both tables highlight countries with minimal total financial
    allocations and commitments.
  \item
    Despite low overall contributions, some countries still provide
    notable military aid.
  \end{itemize}
\item
  \textbf{Lowest Allocators}:

  \begin{itemize}
  \tightlist
  \item
    These countries have not allocated significant funds to Ukraine in
    terms of total financial aid.
  \item
    Cyprus and Denmark stand out with higher military allocations
    compared to others, despite no total allocation.
  \end{itemize}
\item
  \textbf{Lowest Committers}:

  \begin{itemize}
  \tightlist
  \item
    Similar to the allocators, these countries show minimal total
    commitments.
  \item
    The Czech Republic, although low in total commitment, has a
    substantial military commitment.
  \item
    Estonia has a notable total bilateral commitment, suggesting focused
    aid despite a low overall commitment.
  \end{itemize}
\item
  \textbf{Implications}:

  \begin{itemize}
  \tightlist
  \item
    These tables help identify the countries with the least financial
    engagement in supporting Ukraine.
  \item
    The focus on military aid indicates strategic contributions, even
    from countries with otherwise low financial involvement.
  \end{itemize}
\end{itemize}

\bibliography{Tex/ref}





\end{document}
