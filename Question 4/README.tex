% Options for packages loaded elsewhere
\PassOptionsToPackage{unicode}{hyperref}
\PassOptionsToPackage{hyphens}{url}
%
\documentclass[
  12pt,
]{elsarticle}
\usepackage{amsmath,amssymb}
\usepackage{setspace}
\usepackage{iftex}
\ifPDFTeX
  \usepackage[T1]{fontenc}
  \usepackage[utf8]{inputenc}
  \usepackage{textcomp} % provide euro and other symbols
\else % if luatex or xetex
  \usepackage{unicode-math} % this also loads fontspec
  \defaultfontfeatures{Scale=MatchLowercase}
  \defaultfontfeatures[\rmfamily]{Ligatures=TeX,Scale=1}
\fi
\usepackage{lmodern}
\ifPDFTeX\else
  % xetex/luatex font selection
\fi
% Use upquote if available, for straight quotes in verbatim environments
\IfFileExists{upquote.sty}{\usepackage{upquote}}{}
\IfFileExists{microtype.sty}{% use microtype if available
  \usepackage[]{microtype}
  \UseMicrotypeSet[protrusion]{basicmath} % disable protrusion for tt fonts
}{}
\makeatletter
\@ifundefined{KOMAClassName}{% if non-KOMA class
  \IfFileExists{parskip.sty}{%
    \usepackage{parskip}
  }{% else
    \setlength{\parindent}{0pt}
    \setlength{\parskip}{6pt plus 2pt minus 1pt}}
}{% if KOMA class
  \KOMAoptions{parskip=half}}
\makeatother
\usepackage{xcolor}
\usepackage{color}
\usepackage{fancyvrb}
\newcommand{\VerbBar}{|}
\newcommand{\VERB}{\Verb[commandchars=\\\{\}]}
\DefineVerbatimEnvironment{Highlighting}{Verbatim}{commandchars=\\\{\}}
% Add ',fontsize=\small' for more characters per line
\usepackage{framed}
\definecolor{shadecolor}{RGB}{248,248,248}
\newenvironment{Shaded}{\begin{snugshade}}{\end{snugshade}}
\newcommand{\AlertTok}[1]{\textcolor[rgb]{0.94,0.16,0.16}{#1}}
\newcommand{\AnnotationTok}[1]{\textcolor[rgb]{0.56,0.35,0.01}{\textbf{\textit{#1}}}}
\newcommand{\AttributeTok}[1]{\textcolor[rgb]{0.13,0.29,0.53}{#1}}
\newcommand{\BaseNTok}[1]{\textcolor[rgb]{0.00,0.00,0.81}{#1}}
\newcommand{\BuiltInTok}[1]{#1}
\newcommand{\CharTok}[1]{\textcolor[rgb]{0.31,0.60,0.02}{#1}}
\newcommand{\CommentTok}[1]{\textcolor[rgb]{0.56,0.35,0.01}{\textit{#1}}}
\newcommand{\CommentVarTok}[1]{\textcolor[rgb]{0.56,0.35,0.01}{\textbf{\textit{#1}}}}
\newcommand{\ConstantTok}[1]{\textcolor[rgb]{0.56,0.35,0.01}{#1}}
\newcommand{\ControlFlowTok}[1]{\textcolor[rgb]{0.13,0.29,0.53}{\textbf{#1}}}
\newcommand{\DataTypeTok}[1]{\textcolor[rgb]{0.13,0.29,0.53}{#1}}
\newcommand{\DecValTok}[1]{\textcolor[rgb]{0.00,0.00,0.81}{#1}}
\newcommand{\DocumentationTok}[1]{\textcolor[rgb]{0.56,0.35,0.01}{\textbf{\textit{#1}}}}
\newcommand{\ErrorTok}[1]{\textcolor[rgb]{0.64,0.00,0.00}{\textbf{#1}}}
\newcommand{\ExtensionTok}[1]{#1}
\newcommand{\FloatTok}[1]{\textcolor[rgb]{0.00,0.00,0.81}{#1}}
\newcommand{\FunctionTok}[1]{\textcolor[rgb]{0.13,0.29,0.53}{\textbf{#1}}}
\newcommand{\ImportTok}[1]{#1}
\newcommand{\InformationTok}[1]{\textcolor[rgb]{0.56,0.35,0.01}{\textbf{\textit{#1}}}}
\newcommand{\KeywordTok}[1]{\textcolor[rgb]{0.13,0.29,0.53}{\textbf{#1}}}
\newcommand{\NormalTok}[1]{#1}
\newcommand{\OperatorTok}[1]{\textcolor[rgb]{0.81,0.36,0.00}{\textbf{#1}}}
\newcommand{\OtherTok}[1]{\textcolor[rgb]{0.56,0.35,0.01}{#1}}
\newcommand{\PreprocessorTok}[1]{\textcolor[rgb]{0.56,0.35,0.01}{\textit{#1}}}
\newcommand{\RegionMarkerTok}[1]{#1}
\newcommand{\SpecialCharTok}[1]{\textcolor[rgb]{0.81,0.36,0.00}{\textbf{#1}}}
\newcommand{\SpecialStringTok}[1]{\textcolor[rgb]{0.31,0.60,0.02}{#1}}
\newcommand{\StringTok}[1]{\textcolor[rgb]{0.31,0.60,0.02}{#1}}
\newcommand{\VariableTok}[1]{\textcolor[rgb]{0.00,0.00,0.00}{#1}}
\newcommand{\VerbatimStringTok}[1]{\textcolor[rgb]{0.31,0.60,0.02}{#1}}
\newcommand{\WarningTok}[1]{\textcolor[rgb]{0.56,0.35,0.01}{\textbf{\textit{#1}}}}
\usepackage{graphicx}
\makeatletter
\def\maxwidth{\ifdim\Gin@nat@width>\linewidth\linewidth\else\Gin@nat@width\fi}
\def\maxheight{\ifdim\Gin@nat@height>\textheight\textheight\else\Gin@nat@height\fi}
\makeatother
% Scale images if necessary, so that they will not overflow the page
% margins by default, and it is still possible to overwrite the defaults
% using explicit options in \includegraphics[width, height, ...]{}
\setkeys{Gin}{width=\maxwidth,height=\maxheight,keepaspectratio}
% Set default figure placement to htbp
\makeatletter
\def\fps@figure{htbp}
\makeatother
\setlength{\emergencystretch}{3em} % prevent overfull lines
\providecommand{\tightlist}{%
  \setlength{\itemsep}{0pt}\setlength{\parskip}{0pt}}
\setcounter{secnumdepth}{5}
\usepackage{booktabs}
\usepackage{longtable}
\usepackage{array}
\usepackage{multirow}
\usepackage{wrapfig}
\usepackage{float}
\usepackage{colortbl}
\usepackage{pdflscape}
\usepackage{tabu}
\usepackage{threeparttable}
\usepackage{threeparttablex}
\usepackage[normalem]{ulem}
\usepackage{makecell}
\usepackage{xcolor}
\ifLuaTeX
  \usepackage{selnolig}  % disable illegal ligatures
\fi
\IfFileExists{bookmark.sty}{\usepackage{bookmark}}{\usepackage{hyperref}}
\IfFileExists{xurl.sty}{\usepackage{xurl}}{} % add URL line breaks if available
\urlstyle{same}
\hypersetup{
  pdftitle={A Historical Analysis of the Olympic Games},
  hidelinks,
  pdfcreator={LaTeX via pandoc}}

\title{A Historical Analysis of the Olympic Games}
\author{}
\date{\vspace{-2.5em}}

\begin{document}
\maketitle
\begin{abstract}
This article aims to discuss interesting historical facts regarding the
Summer and Winter Olympics, backed by evidence
\end{abstract}

\setstretch{1.2}
\hypertarget{analysing-the-total-metals-won-by-india-and-other-countries}{%
\section{Analysing the Total Metals Won by India and other
Countries}\label{analysing-the-total-metals-won-by-india-and-other-countries}}

\begin{Shaded}
\begin{Highlighting}[]
\NormalTok{table\_total\_metals }\OtherTok{\textless{}{-}} \FunctionTok{total\_metals\_table}\NormalTok{(filtered\_summer\_olympics)}
\end{Highlighting}
\end{Shaded}

\begingroup\fontsize{11pt}{12pt}\selectfont
\begin{longtable}{lrrrr}
\caption{Total Medals by Country \label{tab:total_medals}} \\ 
  \toprule
Country & Bronze & Gold & Silver & Total\_Medals \\ 
  \midrule
Argentina &  28 &  18 &  24 &  70 \\ 
  Brazil &  55 &  23 &  29 & 107 \\ 
  Chile &   4 &   2 &   7 &  13 \\ 
  China & 119 & 170 & 141 & 430 \\ 
  Colombia &  11 &   2 &   6 &  19 \\ 
  India &  11 &   9 &   6 &  26 \\ 
  Indonesia &  11 &   5 &  11 &  27 \\ 
  Mexico &  27 &  13 &  20 &  60 \\ 
  Nigeria &  12 &   2 &   9 &  23 \\ 
  Peru &   0 &   1 &   3 &   4 \\ 
  Russia & 138 & 125 & 113 & 376 \\ 
  South Africa &  27 &  23 &  26 &  76 \\ 
  Turkey &  23 &  38 &  25 &  86 \\ 
  Uruguay &   6 &   2 &   2 &  10 \\ 
   \bottomrule
\end{longtable}
\endgroup

\hypertarget{analysing-the-top-countries-dominating-both-the-summer-and-winter-olympics-over-the-years-a-time-series-analysis}{%
\section{Analysing the Top Countries Dominating both the Summer and
Winter Olympics over the Years: A Time-Series
Analysis}\label{analysing-the-top-countries-dominating-both-the-summer-and-winter-olympics-over-the-years-a-time-series-analysis}}

\hypertarget{analysis-of-total-medals-over-the-years}{%
\subsection{Analysis of Total Medals Over the
Years}\label{analysis-of-total-medals-over-the-years}}

\begin{itemize}
\item
  The plot displays the total number of medals won by the top 5
  countries (GER, SWE, URS, USA) over the years.
\item
  The data spans from the early 1900s to 2010, showing fluctuations in
  medal counts for each country.
\item
  The Soviet Union (URS) exhibits a significant peak in medals won
  around the 1980s, indicating a period of dominance in the Olympics.
\item
  The United States (USA) consistently maintains high medal counts
  throughout the years, with noticeable peaks and troughs.
\item
  Germany (GER) shows an upward trend in medal counts post-1950s, while
  Sweden (SWE) remains relatively stable with lower medal counts
  compared to other countries.
\end{itemize}

+The plot highlights the changing dynamics and competitive landscape of
the Olympics, with different countries emerging as dominant forces in
different time periods.

\begin{figure}

{\centering \includegraphics{README_files/figure-latex/unnamed-chunk-1-1} 

}

\caption{The Top Countries that Dominate in both the Summer and Winter Olympics\label{Figure1}}\label{fig:unnamed-chunk-1}
\end{figure}

\hypertarget{the-most-unlikely-top-countries-as-per-gdp-per-capita-or-by-population}{%
\section{The Most Unlikely Top Countries as per GDP Per Capita or by
Population}\label{the-most-unlikely-top-countries-as-per-gdp-per-capita-or-by-population}}

\textbf{Top 5 Countries by Medals Per Capita}:

\begin{itemize}
\tightlist
\item
  \textbf{URS (Union of Soviet Socialist Republics)}:
\end{itemize}

Stands out significantly, indicating that the country had a high medal
count relative to its population size.

\begin{itemize}
\tightlist
\item
  \textbf{GDR (German Democratic Republic)}:
\end{itemize}

Also shows a strong performance, punching above its weight in terms of
population size. East Germany, officially known as the German Democratic
Republic, was a country in Central Europe from its formation on 7
October 1949 until its reunification with West Germany on 3 October
1990.

\begin{itemize}
\tightlist
\item
  \textbf{ROU (Romania)}:
\end{itemize}

Shows a commendable number of medals won per capita.

\begin{itemize}
\tightlist
\item
  \textbf{FRG (Federal Republic of Germany)}:
\end{itemize}

Consistently performs well in the Olympic Games, and is referencing West
Germany. West Germany{[}a{]} is the common English name for the Federal
Republic of Germany (FRG){[}b{]} from its formation on 23 May 1949 until
the reunification with East Germany on 3 October 1990.

\begin{itemize}
\tightlist
\item
  \textbf{TCH (Czechoslovakia)}:
\end{itemize}

Displays strong performance relative to its population.

\begin{figure}

{\centering \includegraphics{README_files/figure-latex/unnamed-chunk-2-1} 

}

\caption{The Top Countries that Dominate Winning Medals Per Capita\label{Figure2}}\label{fig:unnamed-chunk-2-1}
\end{figure}
\begin{figure}

{\centering \includegraphics{README_files/figure-latex/unnamed-chunk-2-2} 

}

\caption{The Top Countries that Dominate Winning Medals Per Capita\label{Figure2}}\label{fig:unnamed-chunk-2-2}
\end{figure}

\textbf{Top 5 Countries by Medals Per GDP}:

\begin{itemize}
\tightlist
\item
  \textbf{URS (Union of Soviet Socialist Republics)}:
\end{itemize}

Again leads, showing exceptional performance relative to its economic
size.

\begin{itemize}
\tightlist
\item
  \textbf{GDR (German Democratic Republic)}:
\end{itemize}

Demonstrates effective use of economic resources in achieving Olympic
success.

\begin{itemize}
\tightlist
\item
  \textbf{ROU (Romania)}:
\end{itemize}

Efficiently converts its economic resources into medals. - \textbf{FRG
(Federal Republic of Germany)}:

Maintains a strong position in terms of medals per GDP.

\begin{itemize}
\tightlist
\item
  \textbf{CUB (Cuba)}:
\end{itemize}

Punches above its weight economically in terms of Olympic success.

\begin{figure}

{\centering \includegraphics{README_files/figure-latex/unnamed-chunk-3-1} 

}

\caption{The Top Countries that Dominate Winning Medals Per Capita\label{Figure3}}\label{fig:unnamed-chunk-3-1}
\end{figure}
\begin{figure}

{\centering \includegraphics{README_files/figure-latex/unnamed-chunk-3-2} 

}

\caption{The Top Countries that Dominate Winning Medals Per Capita\label{Figure3}}\label{fig:unnamed-chunk-3-2}
\end{figure}

\textbf{Similarity in Figures}: + The two figures show similar countries
leading in both medals per capita and medals per GDP. This similarity
suggests that these countries not only excel in their ability to convert
population size into Olympic success but also manage their economic
resources efficiently to achieve high medal counts.

\begin{itemize}
\item
  The dominance of URS and GDR in both figures highlights their
  historical strength in Olympic competitions during the periods they
  existed, indicating state-sponsored sports programs and a high
  prioritization of athletic success.
\item
  Countries like Romania (ROU) and the Federal Republic of Germany (FRG)
  demonstrate consistent performance, reflecting well-organized sports
  programs that effectively harness both population and economic
  resources.
\end{itemize}

\hypertarget{an-analysis-of-archery}{%
\section{An Analysis of Archery}\label{an-analysis-of-archery}}

+Archery's history at the Olympic Games is split into two periods: the
early era and the modern era.

+The sport featured on the programme of the Olympic Games in 1900, 1904,
1908 and 1920 during the early era. It was also one of the first sports
to feature women's events, in 1904. The competition formats were
inconsistent, often based on local rules, and archery was subsequently
dropped from the programme. World Archery was founded in 1931 with the
goal of rejoining the Games.

+Archery returned to the Olympic Games in 1972 and has remained on the
programme ever since. During this modern era, the competition format has
evolved toward exciting, easily accessible and broadcast-friendly
head-to-head matchplay. Two gold medals, for the individual events, were
awarded from 1972 to 1984; team events were added in 1988.

\begin{figure}

{\centering \includegraphics{README_files/figure-latex/unnamed-chunk-4-1} 

}

\caption{The Top Individual Events by Country in Winning Medals in Archery\label{Figure4}}\label{fig:unnamed-chunk-4}
\end{figure}

\hypertarget{list-of-countries-in-the-figure-above}{%
\subsection{List of Countries in the Figure
above}\label{list-of-countries-in-the-figure-above}}

\begin{itemize}
\tightlist
\item
  AUS: Australia
\item
  CHN: China
\item
  EUN: Unified Team
\item
  FIN: Finland
\item
  FRA: France
\item
  GBR: Great Britain
\item
  ITA: Italy
\item
  JPN: Japan
\item
  KOR: South Korea
\item
  MEX: Mexico
\item
  NED: Netherlands
\item
  POL: Poland
\item
  RUS: Russia
\item
  SWE: Sweden
\item
  UKR: Ukraine
\item
  URS: Union of Soviet Socialist Republics
\item
  USA: United States
\end{itemize}

\begin{figure}

{\centering \includegraphics{README_files/figure-latex/unnamed-chunk-5-1} 

}

\caption{A Pie Chart showcasing the Distibution of Medal Winnings by the Top 5 Countries in Individual Archery Events\label{Figure5}}\label{fig:unnamed-chunk-5}
\end{figure}

\begin{itemize}
\tightlist
\item
  The pie chart above shows the medal distribution by the top 5
  countries in Archery individual events between the years 1900 and
  2012. South Korea (KOR) dominates the chart with 46.7\% of the total
  medals, followed by the USA (20\%), the Soviet Union (URS) with
  15.6\%, Japan (JPN) with 8.9\%, and China (CHN) with 8.9\%.
\end{itemize}

\hypertarget{analysing-the-performance-in-individual-archery-events-by-gender}{%
\subsection{Analysing the Performance in Individual Archery Events by
Gender}\label{analysing-the-performance-in-individual-archery-events-by-gender}}

\begin{itemize}
\item
  The multi-panel plot compares the performance in Archery individual
  events by gender within the top 5 countries (China, Finland, Japan,
  South Korea, Soviet Union, and the USA) from 1970 to 2010.
\item
  Each panel represents one of the top countries, showing the total
  medals won by men and women over the years.
\end{itemize}

+South Korea (KOR) and the USA display noticeable differences in
performance between genders, while other countries like China, Finland,
and Japan show consistent performance between men and women.

\begin{figure}

{\centering \includegraphics{README_files/figure-latex/unnamed-chunk-6-1} 

}

\caption{The Top Countries that Dominate in both the Summer and Winter Olympics\label{Figure6}}\label{fig:unnamed-chunk-6}
\end{figure}

\begin{verbatim}
## # A tibble: 2 x 5
##   term        estimate std.error statistic p.value
##   <chr>          <dbl>     <dbl>     <dbl>   <dbl>
## 1 (Intercept)   -0.167     0.410    -0.408   0.683
## 2 GenderWomen   -0.325     0.561    -0.580   0.562
\end{verbatim}

\begin{figure}

{\centering \includegraphics{README_files/figure-latex/unnamed-chunk-7-1} 

}

\caption{The Top Countries that Dominate in both the Summer and Winter Olympics\label{Figure7}}\label{fig:unnamed-chunk-7}
\end{figure}

\begin{itemize}
\item
  Finally, \ref{Figure6}, determines if there is a correlation between
  gender and the probability of winning a medal by fitted a logistic
  regression model.
\item
  This model helped us predict the probability of winning a gold medal
  based on gender.
\item
  The x-axis represents the gender (Men and Women).
\item
  The y-axis represents the predicted probability of winning a gold
  medal.
\item
  Each dot represents a predicted probability from the logistic
  regression model.
\item
  The predicted probabilities for men are clustered around 0.46, while
  for women, they are around 0.38. This indicates that, historically,
  men have had a higher probability of winning gold medals in Archery
  individual events compared to women within the top 10 countries.
\end{itemize}

\end{document}
