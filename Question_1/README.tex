\documentclass[11pt,preprint, authoryear]{elsarticle}

\usepackage{lmodern}
%%%% My spacing
\usepackage{setspace}
\setstretch{1.2}
\DeclareMathSizes{12}{14}{10}{10}

% Wrap around which gives all figures included the [H] command, or places it "here". This can be tedious to code in Rmarkdown.
\usepackage{float}
\let\origfigure\figure
\let\endorigfigure\endfigure
\renewenvironment{figure}[1][2] {
    \expandafter\origfigure\expandafter[H]
} {
    \endorigfigure
}

\let\origtable\table
\let\endorigtable\endtable
\renewenvironment{table}[1][2] {
    \expandafter\origtable\expandafter[H]
} {
    \endorigtable
}


\usepackage{ifxetex,ifluatex}
\usepackage{fixltx2e} % provides \textsubscript
\ifnum 0\ifxetex 1\fi\ifluatex 1\fi=0 % if pdftex
  \usepackage[T1]{fontenc}
  \usepackage[utf8]{inputenc}
\else % if luatex or xelatex
  \ifxetex
    \usepackage{mathspec}
    \usepackage{xltxtra,xunicode}
  \else
    \usepackage{fontspec}
  \fi
  \defaultfontfeatures{Mapping=tex-text,Scale=MatchLowercase}
  \newcommand{\euro}{€}
\fi

\usepackage{amssymb, amsmath, amsthm, amsfonts}

\def\bibsection{\section*{References}} %%% Make "References" appear before bibliography


\usepackage[round]{natbib}

\usepackage{longtable}
\usepackage[margin=2.3cm,bottom=2cm,top=2.5cm, includefoot]{geometry}
\usepackage{fancyhdr}
\usepackage[bottom, hang, flushmargin]{footmisc}
\usepackage{graphicx}
\numberwithin{equation}{section}
\numberwithin{figure}{section}
\numberwithin{table}{section}
\setlength{\parindent}{0cm}
\setlength{\parskip}{1.3ex plus 0.5ex minus 0.3ex}
\usepackage{textcomp}
\renewcommand{\headrulewidth}{0.2pt}
\renewcommand{\footrulewidth}{0.3pt}

\usepackage{array}
\newcolumntype{x}[1]{>{\centering\arraybackslash\hspace{0pt}}p{#1}}

%%%%  Remove the "preprint submitted to" part. Don't worry about this either, it just looks better without it:
\makeatletter
\def\ps@pprintTitle{%
  \let\@oddhead\@empty
  \let\@evenhead\@empty
  \let\@oddfoot\@empty
  \let\@evenfoot\@oddfoot
}
\makeatother

 \def\tightlist{} % This allows for subbullets!

\usepackage{hyperref}
\hypersetup{breaklinks=true,
            bookmarks=true,
            colorlinks=true,
            citecolor=blue,
            urlcolor=blue,
            linkcolor=blue,
            pdfborder={0 0 0}}


% The following packages allow huxtable to work:
\usepackage{siunitx}
\usepackage{multirow}
\usepackage{hhline}
\usepackage{calc}
\usepackage{tabularx}
\usepackage{booktabs}
\usepackage{caption}


\newenvironment{columns}[1][]{}{}

\newenvironment{column}[1]{\begin{minipage}{#1}\ignorespaces}{%
\end{minipage}
\ifhmode\unskip\fi
\aftergroup\useignorespacesandallpars}

\def\useignorespacesandallpars#1\ignorespaces\fi{%
#1\fi\ignorespacesandallpars}

\makeatletter
\def\ignorespacesandallpars{%
  \@ifnextchar\par
    {\expandafter\ignorespacesandallpars\@gobble}%
    {}%
}
\makeatother

\newenvironment{CSLReferences}[2]{%
}

\urlstyle{same}  % don't use monospace font for urls
\setlength{\parindent}{0pt}
\setlength{\parskip}{6pt plus 2pt minus 1pt}
\setlength{\emergencystretch}{3em}  % prevent overfull lines
\setcounter{secnumdepth}{5}

%%% Use protect on footnotes to avoid problems with footnotes in titles
\let\rmarkdownfootnote\footnote%
\def\footnote{\protect\rmarkdownfootnote}
\IfFileExists{upquote.sty}{\usepackage{upquote}}{}

%%% Include extra packages specified by user

%%% Hard setting column skips for reports - this ensures greater consistency and control over the length settings in the document.
%% page layout
%% paragraphs
\setlength{\baselineskip}{12pt plus 0pt minus 0pt}
\setlength{\parskip}{12pt plus 0pt minus 0pt}
\setlength{\parindent}{0pt plus 0pt minus 0pt}
%% floats
\setlength{\floatsep}{12pt plus 0 pt minus 0pt}
\setlength{\textfloatsep}{20pt plus 0pt minus 0pt}
\setlength{\intextsep}{14pt plus 0pt minus 0pt}
\setlength{\dbltextfloatsep}{20pt plus 0pt minus 0pt}
\setlength{\dblfloatsep}{14pt plus 0pt minus 0pt}
%% maths
\setlength{\abovedisplayskip}{12pt plus 0pt minus 0pt}
\setlength{\belowdisplayskip}{12pt plus 0pt minus 0pt}
%% lists
\setlength{\topsep}{10pt plus 0pt minus 0pt}
\setlength{\partopsep}{3pt plus 0pt minus 0pt}
\setlength{\itemsep}{5pt plus 0pt minus 0pt}
\setlength{\labelsep}{8mm plus 0mm minus 0mm}
\setlength{\parsep}{\the\parskip}
\setlength{\listparindent}{\the\parindent}
%% verbatim
\setlength{\fboxsep}{5pt plus 0pt minus 0pt}



\begin{document}



\begin{frontmatter}  %

\title{Analaysing the Persistence and Patterns of Baby Naming
Conventions}

% Set to FALSE if wanting to remove title (for submission)




\author[Add1]{Annika Pretorius}
\ead{}





\address[Add1]{Stellenbosch University}


\begin{abstract}
\small{
In this study, we analyze historical baby naming trends in the US from
1910 to 2014, focusing on the influence of cultural factors such as the
media and celebrities. Our findings provide insights into the
persistence of popular names and the impact of societal trends on naming
conventions, aiding a toy design agency in selecting resonant character
names for their products.
}
\end{abstract}

\vspace{1cm}





\vspace{0.5cm}

\end{frontmatter}

\setcounter{footnote}{0}



%________________________
% Header and Footers
%%%%%%%%%%%%%%%%%%%%%%%%%%%%%%%%%
\pagestyle{fancy}
\chead{}
\rhead{}
\lfoot{}
\rfoot{\footnotesize Page \thepage}
\lhead{}
%\rfoot{\footnotesize Page \thepage } % "e.g. Page 2"
\cfoot{}

%\setlength\headheight{30pt}
%%%%%%%%%%%%%%%%%%%%%%%%%%%%%%%%%
%________________________

\headsep 35pt % So that header does not go over title




\newpage

\hypertarget{analaysing-the-persistence-and-patterns-of-baby-naming-conventions}{%
\section{Analaysing the Persistence and Patterns of Baby Naming
Conventions}\label{analaysing-the-persistence-and-patterns-of-baby-naming-conventions}}

The naming trends of children have long been a subject of fascination
and study. Recently, a New York-based kids' toy design agency has
expressed interest in leveraging data analytics to understand these
trends better. The primary goal is to identify the factors that
influence baby names, such as popular movie characters, US presidential
candidates, celebrities, and top-ranking songs or artists. By
comprehending these patterns, the agency aims to predict which character
names might resonate best with children, thereby enhancing the appeal of
their toys.

To facilitate this analysis, we have been provided with an extensive
dataset of baby names in the US, spanning from 1910 to 2014. This
dataset includes annual data for all US states, offering a comprehensive
view of naming patterns over more than a century. Additionally, we have
access to population data by state and city, which will help us generate
proportional insights at the state level. To supplement our primary
dataset, we have also compiled data on music, movies, and TV series,
including the Top 100 Billboard songs since 1958 and HBO movie and
series titles with their audience popularity scores.

One of the key aspects of our analysis involves examining the longevity
and persistence of baby names. Specifically, we aim to determine whether
popular names tend to persist over time or if they fade away as fleeting
trends. A useful method for this investigation is the Spearman rank
correlation, which will help us assess the rank similarity of the top 25
most popular boys' and girls' names from one year to the next three
years. This will provide insights into whether the trends of popular
names have become more or less persistent over the decades, particularly
since the 1990s.

Figure \ref{Figure1} below shows the total count of baby names per year
by gender. The blue line represents male names, while the red line
represents female names. This visualization highlights the fluctuations
in the popularity of baby names over time, offering a foundation for our
deeper analysis of naming trends.

\begin{figure}

{\centering \includegraphics{README_files/figure-latex/unnamed-chunk-1-1} 

}

\caption{Total Count of Baby Names by Gender.\label{Figure1}}\label{fig:unnamed-chunk-1}
\end{figure}

Figure \ref{Figure2} below illustrates the proportion of the top 25 baby
names over time for both female and male names.The horizontal axis
represents the years from 1950 to 2014. The vertical axis represents the
proportion of babies given a specific name each year. This is calculated
as the count of babies with that name divided by the total count of
babies for that year.Each line represents a specific name that was in
the top 25 for that year. The color of each line corresponds to a
different name. The plot is divided into two facets: one for female
names and one for male names. This separation helps to compare the
trends for female and male names independently.

\begin{itemize}
\item
  Female Names: The left facet shows the proportion trends for the top
  25 female names over time. The plot indicates that the popularity of
  the top names was relatively higher in the past (1950s to 1980s) and
  has become more evenly distributed in recent years.
\item
  Male Names: The right facet shows the proportion trends for the top 25
  male names over time. Similar to female names, the plot indicates that
  the popularity of the top names was higher in the past and has become
  more evenly distributed in recent years.
\end{itemize}

The peaks in the plot indicate years when specific names were
particularly popular. The valleys indicate years when the popularity of
those names decreased. Declining Trends: For both female and male names,
the plot shows a general declining trend in the proportion values,
indicating that the popularity of top names has become less concentrated
over time. This suggests a diversification in naming trends, with a
wider variety of names being used in recent years.

\begin{figure}

{\centering \includegraphics{README_files/figure-latex/unnamed-chunk-2-1} 

}

\caption{The Proportion of the Top 25 Baby Names and their Persistence Over Time\label{Figure2}}\label{fig:unnamed-chunk-2}
\end{figure}

\hypertarget{analysing-the-persistence-of-names-over-time-a-time-series-analysis}{%
\section{Analysing the Persistence of Names over Time: A Time Series
Analysis}\label{analysing-the-persistence-of-names-over-time-a-time-series-analysis}}

Figure \ref{Figure3} below showcases the persistence of the most popular
baby names from a given dataset. The color gradient from white to blue
represents the presence of a name in the top 25 baby names for each
year, with darker shades of blue indicating higher persistence. Notable
names such as ``Jessica,'' ``Jennifer,'' and ``Emily'' show strong
persistence over the years, particularly in the 1980s and 1990s. Names
like ``Susan'' and ``Mary'' were more popular in earlier decades, while
``Ashley'' and ``Amanda'' gained popularity later on.Names like
``Robert,'' ``Michael,'' and ``John'' demonstrate strong persistence,
especially from the 1960s to the 2000s. Other names such as ``Matthew,''
``Joshua,'' and ``Joseph'' show high presence in the top 25 names in the
1980s and 1990s. Male names show more consistent presence over a longer
period compared to female names, which exhibit more variation and
changes over the years.

\begin{verbatim}
## [1] "Data before pivoting:"
## # A tibble: 6 x 5
##    Year Name    Proportion  Rank Gender
##   <dbl> <chr>        <dbl> <int> <chr> 
## 1  1991 Jessica    0.00426     1 F     
## 2  1987 Jessica    0.00448     1 F     
## 3  1990 Jessica    0.00400     1 F     
## 4  1989 Jessica    0.00405     1 F     
## 5  1988 Jessica    0.00417     1 F     
## 6  1992 Jessica    0.00397     1 F
\end{verbatim}

\begin{figure}

{\centering \includegraphics{README_files/figure-latex/unnamed-chunk-3-1} 

}

\caption{The Proportion of the Top 25 Baby Names and their Persistence Over Time for both Genders\label{Figure3}}\label{fig:unnamed-chunk-3}
\end{figure}

\hypertarget{analyzing-how-long-each-name-stayed-in-top-25}{%
\section{Analyzing how long each name stayed in top
25}\label{analyzing-how-long-each-name-stayed-in-top-25}}

\begin{figure}

{\centering \includegraphics{README_files/figure-latex/unnamed-chunk-4-1} 

}

\caption{The Number of Years the Top 25 Baby Names Persisted Over Time for both Genders\label{Figure4}}\label{fig:unnamed-chunk-4}
\end{figure}

We can therefore see that the top boy names over the last few decades
illustrate significant persistence in comparison to that of girl names,
therefore, we will further analyse factors influencing girl names as
they display significantly higher variability which could be linked to
alternate factors.

\hypertarget{analysing-the-hbo-dataset}{%
\section{Analysing the HBO Dataset}\label{analysing-the-hbo-dataset}}

\begin{figure}

{\centering \includegraphics{README_files/figure-latex/unnamed-chunk-5-1} 

}

\caption{The Persistence of HBO names in the top rankings over time\label{Figure5}}\label{fig:unnamed-chunk-5}
\end{figure}

We can see the above persistence is all male names within data regarding
movies on HBO over the last few decades.

\hypertarget{the-persistence-of-hbo-names-in-the-top-rankings-over-time.}{%
\section{The Persistence of HBO names in the Top Rankings over
Time.}\label{the-persistence-of-hbo-names-in-the-top-rankings-over-time.}}

In the Figure \ref{Figure6} below, the name ``John'' stands out as the
most persistent name, having stayed in the top rankings for 40 years.
This indicates that ``John'' was a popular name in HBO shows over a
prolonged period. Names like ``Robert,'' ``Michael,'' ``David,'' and
``Richard'' also exhibit high persistence, each maintaining their
presence in the top rankings for over 20 years.The presence of these
names over many years suggests a trend in character naming conventions
in HBO shows. The recurrence of these names might reflect cultural
preferences or character archetypes frequently used in HBO's
storytelling.

The distribution of the years in top rankings shows that while some
names had a very long presence, there is a gradual decline as we move
down the list. Names like ``Tom,'' ``Peter,'' and ``Jack,'' though still
significant, have a lesser persistence compared to the top names.

The bar chart effectively highlights the relative persistence of each
name, with the length of the bars corresponding to the number of years
each name stayed in the top rankings. The use of horizontal bars makes
it easy to compare the persistence across different names. This analysis
helps in understanding the longevity and popularity of certain names
within HBO shows, providing insights into naming trends and cultural
influences over the years.

\begin{figure}

{\centering \includegraphics{README_files/figure-latex/unnamed-chunk-6-1} 

}

\caption{The Persistence of HBO names in the top rankings over time.\label{Figure6}}\label{fig:unnamed-chunk-6}
\end{figure}

Now i will analyse the correlation between the persistent boy names
within the Baby names dataframe in comparison to that of the most
persistent names within the HBO baby frame, which all happen to be male.

\begin{figure}

{\centering \includegraphics{README_files/figure-latex/unnamed-chunk-7-1} 

}

\caption{The Correlation Between the Persistence of HBO names and Baby Names.\label{Figure7}}\label{fig:unnamed-chunk-7}
\end{figure}

\begin{verbatim}
## [1] "Spearman correlation value:  0.51"
\end{verbatim}

In Figure \ref{Figure7} above the correlation between the persistence of
baby names in the top 25 rankings and HBO names in the top 25 rankings
over time are analysed. Each point on the scatter plot represents a name
that appears in each of the datasets. The x-axis shows the number of
years a name remained in the top 25 baby names, while the y-axis
indicates the number of years the same name stayed in the top 25 HBO
rankings.

A positive Spearman correlation value of 0.509. was calculated, which
indicates a moderate positive correlation between the persistence of
baby names and HBO names. This suggests that names that tend to be
popular for longer periods as baby names are also likely to remain
popular in HBO rankings for a significant duration. The blue line
represents a linear fit, illustrating the trend of this correlation.
While there is some spread in the data, with a number of outliers, the
general trend suggests that there is a relationship between the duration
of name popularity in both contexts.

\bibliography{Tex/ref}





\end{document}
